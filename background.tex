%!TEX root=ndma.tex
\paragraph{Notation.} 
Throughout the paper: $f \colon X \to Y$ is a total function, $f \colon X \inj Y$ is total and injective, $f \colon X \pto Y$ is partial; $\dom(f)$ is the subset of $X$ where $f$ is defined, $\Im(f)$ its image. Symbol $\omega$ denotes the set of natural numbers. For $s$ a sequence, we let $s_i$ or $s(i)$ denote its $i^{\mathit{th}}$ element. $R^*$ is the symmetric, transitive, reflexive closure of binary relation $R$. We use $\circ$ for (partial) function composition and also for ``relational'' composition, as usual, by seeing functions as relations.

We shall now briefly introduce nominal sets; we refer the reader to \cite{GP02} for more details on the subject. We assume a countable set of \emph{names} $\names$, and we write $\Perm$ for the group of finite-kernel permutations of $\names$, namely those bijections $\pi \colon \names \to \names$ such that the set $\{ a \mid \pi(a) \neq a \}$ is finite.
\begin{definition}
A \emph{nominal set} is a set $X$ along with an action for $\Perm$, that is a function $\cdot \colon \Perm \times X \to X$ such that, for all $x \in X$ and $\pi,\pi' \in \Perm$, $id_\names \cdot x = x$ and $(\pi \circ \pi') \cdot x = \pi \cdot (\pi' \cdot x)$. Also, it is required that each $x \in X$ has \emph{finite support}, meaning that there exists a finite $S \subseteq \names$ such that, for all $\pi \in \Perm$, $\restr{\pi}{S} = id_S$ implies $\pi \cdot x = x$. We denote the least\footnote{It is a theorem that whenever there is a finite support, there is also a least support.} such $S$ with $\supp(x)$. An \emph{equivariant function} from nominal set $X$ to nominal set $Y$ is a function $f : X \to Y$ such that, for all $\pi$ and $x$, $f(\pi \cdot x) = \pi \cdot f(x)$.
\end{definition}
%
\begin{definition}
Given $x \in X$, the \emph{orbit} of $x$, denoted by $\orb(x) $, is the set $\{ \pi \cdot x \mid \pi \in \Perm\} \subseteq X$. For $S \subseteq X$, we write $\orb(S)$ for $\{ \orb(x) \mid x \in S\}$. We call $X$ \emph{orbit-finite} when $\orb(X)$ is finite.
\end{definition}

\noindent Note that $\orb(X)$ is a partition of $X$. The prototypical nominal set is $\names$ with $\pi \cdot a = \pi(a)$ for each $a \in \names$; we have $\supp(a) = \{a\}$, and $\orb(a) = \names$.
