%!TEX root=ndma.tex
%
%
%
An \emph{ultimately periodic} word is a word of the form $uv^\omega$, with $u,v$ finite words. Given a language of infinite words $\Lang$, let $UP(\Lang)$ be its \emph{ultimately periodic fragment} $\{ \alpha \in \Lang \mid \alpha = uv^\omega \land u,v \; \text{are finite} \}$. It has been proven in \cite{CalbrixNP93,Buchi62} that, for every two $\omega$-regular languages $\Lang_1$ and $\Lang_2$, $UP(\Lang_1) = UP(\Lang_2)$ implies $\Lang_1 = \Lang_2$, i.e.\ $\omega$-regular languages are characterised by their ultimately periodic fragments.
In this section we aim to extend this result to the nominal setting. 

The preliminary result to establish, as in the classical case, is that every non-empty nominal $\omega$-regular language $\Lang$ contains at least one ultimately periodic word. For $\omega$-regular languages, this involves finding a loop through accepting states in the automaton and iterating it. For \hdmas, freshness constraints could forbid consuming the same name in consecutive traversals of the same transition.
We first show that, given a loop in a \hdma{}, there always is a path induced by consecutive traversals of the loop, such that its initial and final configurations coincide. Thus, such path can be taken an arbitrary number of times.

Fix a loop $L \;:=\; p_0 \htr{l_0}{\sigma_0} p_1 \htr{l_1}{\sigma_1} \dots \htr{l_{n-1}}{\sigma_{n-1}} p_0$ (the specific \hdma{} is not relevant). We write $\ul{i}$ for $i \mod n$. For all $i=0,\dots,n-1$, let $\widehat{\sigma}_i \colon \weight{p_\ul{i+1}} \pto \weight{p_i}$ be the partial maps telling the history of old registers and ignoring the new ones, formally $\widehat{\sigma}_i := \sigma_i \setminus \{ (x,y) \in \sigma_i \mid y = \star \}$, and let $\widehat{\sigma} \colon \weight{p_0} \pto \weight{p_0}$ be their composition $\widehat{\sigma}_0 \circ \widehat{\sigma}_1 \dots \circ \widehat{\sigma}_{n-1}$. We define the set $I$ as the greatest subset of $\dom(\widehat{\sigma})$ such that $ \widehat{\sigma}(I) = I$,
i.e.\ $I$ are the registers of $p_0$ that ``survive'' along $L$. We denote by $T$ all the other registers, namely $T := \weight{p_0} \setminus I$. These are registers whose content is eventually discarded (not necessarily within a single loop traversal), as the following lemma states.
%
%
\begin{lemma}
\label{lem:rho-forget}
Given any $x \in T$, let $\{x_j\}_{j \in J_x}$ be the smallest sequence that satisfies the following conditions: $x_0 = x$ and $x_{j+1} = \sigma_{\ul{j}}^{-1}(x_j)$, where $j+1 \in J_x$ only if $\sigma_{\ul{j}}^{-1}(x_j)$ is defined. Then $J_x$ has finite cardinality.
\end{lemma}
%
Now, consider any assignment $\rrho_0 \colon \weight{p_0} \to \names$. We give some lemmata about paths that start from $(p_0,\rrho_0)$ and are induced by consecutive traversals of $L$. The first one says that the assignment for $I$ given by $\rrho_0$ is always recovered after a fixed number of traversals of $L$, regardless of which symbols are consumed. In the following, given a sequence of transitions $P$, we write $(q_1,\rho_1) \TrP{v}{P} (q_2,\rho_2)$ whenever $(q_1,\rho_1) \Tr{v} (q_2,\rho_2)$ and such path is induced by $P$.
%
\begin{lemma} 
\label{lem:idI}
There is $\id \geq 1$ such that, for all $v_0,\dots,v_{\id-1}$ satisfying $(p_0,\rrho_0) \TrP{v_0}{L} (p_0,\rrho_1) \TrP{v_1}{L} \dots \TrP{v_{\id-1}}{L} (p_0,\rrho_\id)$ we have $\restr{ \rrho_\id }{I} = \restr{ \rrho_0 }{I}$.
\end{lemma}
%
The second one says that, after a minimum number of traversals of $L$, a configuration can be reached where the initial values of $T$, namely those assigned by $\rrho_0$, cannot be found in any of the registers.
%
\begin{lemma}
There is $\forg \geq 1$ s.t., for all $\gamma \geq \forg$ ,there are $v_0,\dots,v_{\gamma-1}$ satisfying $(p_0,\rrho_0) \TrP{v_0}{L} (p_0,\rrho_1) \TrP{v_1}{L} \dots \TrP{v_{\gamma-1}}{L} (p_0,\rrho_\gamma)$, with $\Im(\rrho_\gamma) \cap \rrho_0(T) = \varnothing$.
\label{lem:forgetT}
\end{lemma}
%
%
We give the dual of the previous lemma: if we start from a configuration where registers are not assigned values in $\rrho_0(T)$, then these values can be assigned back to $T$ in a fixed number of traversals of $L$, regardless of the initial assignment.

\begin{lemma}
There is $\ass \geq 1$ such that,
for any $\trho_0 \colon \weight{p_0} \to \names$ with $\Im(\trho_0) \cap \rrho_0(T) = \varnothing$, there are $v_0,\dots,v_{\ass-1}$ satisfying $ (p_0,\trho_0) \TrP{v_0}{L} (p_0,\trho_1) \TrP{v_1}{L} \dots \TrP{v_{\ass-1}}{L} (p_0,\trho_\ass)$, with $\restr{\trho_\ass}{T} = \restr{\rrho_0}{T}$.
\label{lem:initT}
\end{lemma}
%
%
Finally, we combine the above lemmata. We construct a path where: (1) the values assigned to $T$ are forgotten and then recovered (2) the values assigned to $I$ are swapped, but the initial assignment is periodically regained. Therefore, the length of such path should allow (1) and (2) to ``synchronize'', so that the final assignment is again $\rrho_0$.

\begin{theorem}
\label{thm:loop}
%
For each loop $L$ with initial state $p_0$, and assignment $\rrho_0 \colon \weight{p_0} \to \names$, there are $v_0,\dots,v_n$ such that $(p_0, \rrho_0) \TrP{v_0}{L} (p_0, \rrho_1) \TrP{v_1}{L} \cdots \TrP{v_n}{L} (p_0,\rrho_0) $.
\end{theorem}
%
\begin{example} We justify the construction on the \hdma{} of \cref{fig:upwords-ex}, with initial assignment $\rho_0(x_0) = a$,$\rho_0(y_0) = b$ and $\rho_0(z_0) = c$. Consider the loop $L$ formed by all the depicted transitions. We have $I = \{x_0,y_0\}$ and $T = \{z_0\}$. Look at the path $(q_0,[\subs{a}{x_0},\subs{b}{y_0},\subs{c}{z_0}]) \tr{c} (q_1,[\subs{b}{x_1},\subs{a}{y_1},\subs{c}{z_1}]) \tr{d} (q_2,[\subs{b}{x_2},\subs{a}{y_2},\subs{d}{z_2}])
	\tr{b} (q_0,[\subs{b}{x_0},\subs{a}{y_0},\subs{d}{z_0}])$ where $d \neq a,b,c$. The values of $x_0$ and $y_0$ are swapped according to the permutation $(a \; b)$, and $d$ is assigned to $z_0$. Our aim is to recover $\rho_0$ again. According to \cref{lem:idI}, $x_0$ and $y_0$ get their assignment back in $\theta = 2$ traversals of $L$ (in fact $(a\; b)^2 = (a\; b)$). As for $z_0$, its assignment is established in the second transition, but $c$ should not have been assigned to any register of $q_1$ in order for it to be consumed during this transition. This is where \cref{lem:forgetT} comes into play: it says that in at least $\epsilon = 1$ traversals of $L$ the name $c$ is discarded. This is exactly what happens in the path shown above. Then we can assign $c$ to $z_0$ in another $\zeta = 1$ traversal of $L$, according to \cref{lem:initT}. Since $\epsilon + \zeta  = \theta = 2$, traversing $L$ twice is enough (e.g., consider the path $cdbdca$).
\end{example}
%
\noindent Finally we introduce the main results of this section.
%
\begin{theorem}
\label{thm:up-fragment}
When $\Lang$ is a non-empty nominal $\omega$-regular language, $UP(\Lang) \neq \emptyset$.
\end{theorem}

\begin{theorem}
\label{thm:up-determinacy}
For $\Lang_1,\Lang_2$ nominal $\omega$-regular, $UP(\Lang_1) = UP(\Lang_2) \implies\Lang_1 = \Lang_2$.
\end{theorem}
% 
Note that a similar result could not be achieved in the presence of so-called \emph{global freshness} \cite{Tze11}, e.g.\ the one-state automaton accepting only globally fresh symbols would have empty ultimately periodic fragment, just like the empty language. As a concluding remark, we note that, by \cref{thm:up-determinacy}, every $\omega$-regular language is characterized by a sublanguage of finitely supported words (the support of $uv^\omega$ just contains the finitely many symbols in $u$ and $v$). We find this result appealing, given the central role of the notion of support in the nominal setting. 