%!TEX root=ndma.tex


This paper aims at contributing to the theory of formal verification of \emph{global computing} systems, by extending the theory of Muller automata to the case of infinite alphabets, while retaining decidability. In this way, it is possible to adapt the classical automata-theoretic approach to formal specification and verification \cite{ClarkeS01} to systems with resource generation capabilities, where the number of possible resources is infinite, provided that these systems enjoy a \emph{finite memory} property. 

Transition structures, in the form of automata, are used to represent logic formalisms interpreted over finite and infinite words, dating back to \cite{Buchi60,Elgot61}. The possibility of translating modal logic formulas to automata led to the development of \emph{model checking}. 
%
Systems that feature resource allocation (e.g.\ \cite{MilnerPW92}), typically in the form of \emph{name allocation}, pose specific challenges. For instance, they have ad-hoc notions of bisimulation, which cannot be captured by standard set-theoretic models. Transition structures that correctly model name allocation have been proposed in various forms, including coalgebras over presheaf categories \cite{FioreT01,BonchiBCG11,Miculan08,GhaniYV04,SammartinoM14}, history-dependent automata \cite{MontanariP05}, and automata over nominal sets \cite{BojanczykKL11}. Equivalence of these models has been established both at the level of base categories \cite{GadducciMM06,FioreS06,CianciaKM10} and of coalgebras \cite{CianciaM10}. More recently, the field of \emph{nominal automata} has essentially used the same structures, no longer as semantic models, but rather as acceptors of languages of finite words (see e.g., \cite{Tze11,KST12,GC11,BojanczykKL11}). In particular, the obtained languages are based on infinite alphabets, but still enjoy finite memory (in fact, the well known \emph{register automata} of Francez and Kaminski can be regarded as nominal automata, see \cite{BojanczykKL11}). 


The case of infinite words over nominal alphabets is more problematic, as an infinite word over an infinite alphabet is generally not \emph{finitely supported}.\footnote{The notion of finite support, coming from the theory of nominal sets, will be clarified later; roughly, finitely supported elements just use a finite set of names.} Consider a machine that reads any symbol from an infinite, countable alphabet, and never stores it. Clearly, such a machine has finite (empty) memory. The set of its traces is simply described as the set of all infinite words over the alphabet. However, in the language we have various species of words. Some of them are finitely supported, e.g.\ words that consist of the infinite repetition of a finite word. Some others are not finitely supported, such as the word enumerating all the symbols of the alphabet. Such words lay inherently out of the realm of nominal sets. However, the existence of these words does not give infinite memory to the language. More precisely, words without finite support can not be ``singled out'' by a finite memory machine; if a machine accepts one of them, then it will accept infinitely many others, including finitely supported words.  

This work aims at translating the above intuitions into precise mathematical terms, in order to define a class of languages made of infinite words over infinite alphabets, enjoying finite-memory properties. We extend automata over nominal sets to handle infinite words, by imposing a (Muller-style) acceptance condition 
over the \emph{orbits} (not the states!) of automata. By doing so, it turns out that our languages not only are finite-memory, but they retain computational properties, such as closure under boolean operations and decidability of emptiness (thus, containment and equivalence), which we prove by providing finite representations, and effective constructions. As in the case of standard $\omega$-automata, the shift to infinite words requires these results to be proved from scratch, as it is not possible to merely extend proofs from the finite words case. These results enable automata-theoretic model checking to be performed on systems with infinite resources, using traditional model-checking algorithms. 

Furthermore, we prove that the defined languages are determined by their ultimately-periodic fragments. This theorem is fundamental for \emph{learning} logical properties (see e.g., \cite{MP95}, or \cite{FCCTW08}) and has been used to provide a complete minimisation procedure for equivalent representations of Muller automata \cite{CV12}. Establishing this theoretical result in the nominal case is an important step towards the application of such techniques to global computing scenarios.