\documentclass{elsarticle}

\title{?}
%\thanks{Research partially funded by projects EU 
% ASCENS (nr. 257414), EU QUANTICOL (nr. 600708), 
% IT MIUR CINA and PAR FAS 2007-2013 Regione Toscana TRACE-IT.}
% \thanks{An extended version containing full proofs is available at \cite{arxivCS}}
%}
%\author{Vincenzo Ciancia\inst{1} \and Matteo Sammartino\inst{2}}
%\institute{ISTI-CNR, Pisa \and Dipartimento di Informatica, Universit\`a di Pisa, Pisa }


\usepackage{amsmath}
\usepackage{amsthm}
\usepackage{amssymb}
\usepackage{subfigure}
\usepackage{tikz}
\usepackage{paralist}
\usepackage{aliascnt}
\usetikzlibrary{arrows,automata,positioning,calc,fit}
\newcommand{\subfigureautorefname}{\figureautorefname}
%\usepackage{cleveref}
\usepackage[hidelinks]{hyperref}
\newcommand{\cref}[1]{\autoref{#1}}
\usepackage{todonotes,mathpartir,enumerate,nicefrac}
\usepackage[all]{xy}
\usepackage[utf8]{inputenc}

\SelectTips{cm}{}

\DeclareMathOperator{\dom}{dom}

\newcommand{\todosm}[1]{\todo[size=\tiny]{#1}}

\newcommand{\autom}{A}
\newcommand{\hdma}{hDMA}
\newcommand{\hdmas}{\hdma s}

% Transition arrows
\newcommand{\Trarrow}{\Longrightarrow}
\newcommand{\Tr}[1]{\overset{#1}{\Trarrow}}
\newcommand{\TrP}[2]{\Tr{#1}_{#2}}
\newcommand{\htrind}[3]{\overset{#1}{\underset{#2}{\trarrow_{#3}}}}
\newcommand{\trind}[2]{\overset{#1}{\trarrow_{#2}}}

\newcommand{\pto}{\rightharpoonup}
\newcommand{\ul}[1]{{\underline{#1}}}
\newcommand{\restr}[2]{\ensuremath{\left.#1\right|_{#2}}}
\newcommand{\syncp}{\otimes}

\newcommand{\tstr}{\mathcal{T}}

% Nominal notation
\newcommand{\nomG}{\ensuremath{\mathbf{G}}}
\newcommand{\eqsub}{\subseteq_{eq}}
\newcommand{\reg}[1]{\Vert #1 \Vert}
%\newcommand{\edge}[2]{#1\!-\!#2}

% Relation symbols
\newcommand{\subrel}{\sqsubseteq}

% Graph notation
\newcommand{\edge}[2]{
\begin{tikzpicture}[baseline,node distance=1em]
\tikzstyle{vertex} = [anchor=base,circle,draw=black,fill=none,thick,inner sep=0,outer sep=0,minimum height=2em] 

\node[vertex] (1) {\ensuremath{#1}};
\node[vertex,right=of 1] (2) {\ensuremath{#2}};
\path (1) edge (2);
\end{tikzpicture}
}

% Symbols for ultimately periodic words
\newcommand{\regPerm}{I}
\newcommand{\regForg}{T}
\newcommand{\id}{\theta}
\newcommand{\forg}{\epsilon}
\newcommand{\ass}{\zeta}

\newcommand{\rrho}{\hat{\rho}}
\newcommand{\trho}{\tilde{\rho}}

\newcommand{\subs}[2]{\nicefrac{#1}{#2}}

\newcommand{\lab}[1]{{${\scriptstyle #1}$}}

%% Box
\newcommand{\tbox}[2]{
\medskip
\noindent%
\marginpar{#1}%
\fbox{%
\parbox{\linewidth}{#2}
}%
\medskip
}

\input{vincenzo-macros}


%% Theorems
%%% Fixes for autoref
\newtheorem{dummy}{***}[section] % add [section] for numbering within sections
\newcommand{\mynewtheorem}[2]{
  \newaliascnt{#1}{dummy}
  \newtheorem{#1}[#1]{#2}
  \aliascntresetthe{#1}
  % maybe we will squish some autoref defaults, but who cares?
  \expandafter\def\csname #1autorefname\endcsname{#2}
}


\newcommand{\theoremname}{Theorem}
\newcommand{\lemmaname}{Lemma}
\newcommand{\propositionname}{Proposition}
\newcommand{\definitionname}{Definition}
\newcommand{\examplename}{Example}
\newcommand{\remarkname}{Remark}
\newcommand{\propertyname}{Property}
\newcommand{\notationname}{Notation}
\newcommand{\conventionname}{Convention}

\theoremstyle{plain}
  \mynewtheorem{theorem}{Theorem}
  \mynewtheorem{proposition}{Proposition}
  \mynewtheorem{corollary}{Corollary}
  \mynewtheorem{property}{Property}
  \mynewtheorem{lemma}{Lemma}
  \mynewtheorem{conjecture}{Conjecture}
\theoremstyle{definition}
  \mynewtheorem{definition}{Definition}
  \mynewtheorem{example}{Example}
\theoremstyle{remark}
  \newtheorem*{notation}{\notationname}
  \mynewtheorem{remark}{Remark}
  \mynewtheorem{convention}{Convention}
  \mynewtheorem{invariant}{Invariant}


\raggedbottom
\frenchspacing

\begin{document}

 \begin{abstract}
\end{abstract}

\maketitle

\section{\nomG-nominal deterministic Muller automata}
%!TEX root=main.tex
%We assume an enumerable set of names $\names$, equipped with a relation $R$ \marginnote{Requirements sulla relazione?}.
%
\begin{definition}\label{def:gndma}
 A \emph{deterministic \nomG-nominal automaton} is a tuple $(Q,\tr{},q_0,A)$ where:
 
  \begin{itemize}
  \item $Q$ is an orbit-finite \nomG-nominal set of \emph{states}, with $q_0 \in Q$ the \emph{initial state};
%  
  \item $A \subseteq \Pow(\orb(Q))$ is a set of sets of orbits, intended to be used as an acceptance condition in the style of Muller automata.
%  
 \item $\tr{} \eqsub Q \times \names \times Q$ is an (equivariant) \emph{transition relation}, made up of triples $q_1 \tr{a} q_2$, having \emph{source} $q_1$, \emph{target} $q_2$ and \emph{label} $a \in \nomG$;
%  
\item the transition relation is \emph{deterministic}, that is: for each $q \in Q$ and $a \in \nomG$ there is exactly one transition with source $q$ and label $a$;
%  
%\item the transition relation is \emph{equivariant}, that is, invariant under permutations: there is a transition $q_1 \tr a q_2$ if and only if, for all $\pi$, also the transition $\pi \cdot q_1 \tr{\pi(a)} \pi \cdot q_2$ is present.
\end{itemize}
\end{definition}


\section{HDG-automata}
%!TEX root=main.tex
\paragraph{Notation} Given a relation (e.g.\ a graph) $R$, we write $x \in R$ whenever $x$ belongs to the carrier of $R$, $X \subseteq R$ whenever $x \in R$,, for each $x \in X$, and $R' \subrel R$ whenever $R'$ is a sub-relation of $R$, i.e., $xR'y$ implies $xRy$, for all $x,y \in R'$.

Given a state $q$, $Lab(q)$ are the possible outgoing labels:
\[
	Lab(q) = \reg{q} \cup \{ \star_S \mid S \subseteq \reg{q} \}
\]
where $\star_S$ stands for any fresh vertex connected to those in $S$.
\footnote{
Se il grafo è diretto abbiamo bisogno di una stellina più complicata: $\star_{I,O}$, dove $I$ sono i vertici con nuovi archi verso $\star$ e $O$ quelli con nuovi archi da $\star$.
}

\begin{definition}
\label{hdg}
 A \emph{deterministic history-dependent Muller automaton over \nomG} (\nomG-ndMa) is a tuple $(Q,\reg{\cdot},\htr{}{},q_0,\rho_0,A)$ where:
 
  \begin{itemize}
  \item $Q$ is a finite set of states and $q_0 \in Q$ is the \emph{initial state};
%  
	\item $\reg{\cdot}$ assigns to each state $q \in Q$ a finite graph;
  \item $A \subseteq \Pow(Q)$ is a a Muller accepting set;
%  
 \item $\htr{}{}$ is the transition relation, of the form
\[
	q \htr{\sigma}{l} q'
\]
where $l \in Lab(q)$ and $\sigma \colon \reg{q'} \to \reg{q} \cup \{ l \}$ is a (\emph{history}) map such that, whenever $x \reg{q'} y$:
\begin{itemize} 
	\item if $\sigma(x) \neq \star_S \neq \sigma(y)$ then $\sigma(x) \reg{q} \sigma(y)$;
	\item if $\sigma(x) = \star_S$ then $\sigma(y) \in S$ (and the other way around).
\end{itemize}	
%
\item $\rho_0 \colon \reg{q_0} \to \nomG$ is the \emph{initial assignment} to registers of $q_0$;
%  
\item the transition relation is \emph{deterministic}, that is: for each $q \in Q$ and $l \in Lab(q)$ there is exactly one transition with source $q$ and label $l$;
%  
%\item the transition relation is \emph{equivariant}, that is, invariant under permutations: there is a transition $q_1 \tr a q_2$ if and only if, for all $\pi$, also the transition $\pi \cdot q_1 \tr{\pi(a)} \pi \cdot q_2$ is present.
\end{itemize}	
\end{definition}

\begin{example}
The language of all finite paths from $a$ to $b$.
\[
	\Lang = \{ a v_1 \dots v_n b \mid \edge{a}{b}, \edge{v_i}{v_{i+1}}, \edge{v_n}{b}, 1 \leq i \leq n-1 \}
\]	
\end{example}



%\section{Introduction}\label{sec:introduction}%!TEX root=ndma.tex


This paper aims at contributing to the theory of formal verification of \emph{global computing} systems, by extending the theory of Muller automata to the case of infinite alphabets, while retaining decidability. In this way, it is possible to adapt the classical automata-theoretic approach to formal specification and verification \cite{ClarkeS01} to systems with resource generation capabilities, where the number of possible resources is infinite, provided that these systems enjoy a \emph{finite memory} property. 

Transition structures, in the form of automata, are used to represent logic formalisms interpreted over finite and infinite words, dating back to \cite{Buchi60,Elgot61}. The possibility of translating modal logic formulas to automata led to the development of \emph{model checking}. 
%
Systems that feature resource allocation (e.g.\ \cite{MilnerPW92}), typically in the form of \emph{name allocation}, pose specific challenges. For instance, they have ad-hoc notions of bisimulation, which cannot be captured by standard set-theoretic models. Transition structures that correctly model name allocation have been proposed in various forms, including coalgebras over presheaf categories \cite{FioreT01,BonchiBCG11,Miculan08,GhaniYV04,SammartinoM14}, history-dependent automata \cite{MontanariP05}, and automata over nominal sets \cite{BojanczykKL11}. Equivalence of these models has been established both at the level of base categories \cite{GadducciMM06,FioreS06,CianciaKM10} and of coalgebras \cite{CianciaM10}. More recently, the field of \emph{nominal automata} has essentially used the same structures, no longer as semantic models, but rather as acceptors of languages of finite words (see e.g., \cite{Tze11,KST12,GC11,BojanczykKL11}). In particular, the obtained languages are based on infinite alphabets, but still enjoy finite memory (in fact, the well known \emph{register automata} of Francez and Kaminski can be regarded as nominal automata, see \cite{BojanczykKL11}). 


The case of infinite words over nominal alphabets is more problematic, as an infinite word over an infinite alphabet is generally not \emph{finitely supported}.\footnote{The notion of finite support, coming from the theory of nominal sets, will be clarified later; roughly, finitely supported elements just use a finite set of names.} Consider a machine that reads any symbol from an infinite, countable alphabet, and never stores it. Clearly, such a machine has finite (empty) memory. The set of its traces is simply described as the set of all infinite words over the alphabet. However, in the language we have various species of words. Some of them are finitely supported, e.g.\ words that consist of the infinite repetition of a finite word. Some others are not finitely supported, such as the word enumerating all the symbols of the alphabet. Such words lay inherently out of the realm of nominal sets. However, the existence of these words does not give infinite memory to the language. More precisely, words without finite support can not be ``singled out'' by a finite memory machine; if a machine accepts one of them, then it will accept infinitely many others, including finitely supported words.  

This work aims at translating the above intuitions into precise mathematical terms, in order to define a class of languages made of infinite words over infinite alphabets, enjoying finite-memory properties. We extend automata over nominal sets to handle infinite words, by imposing a (Muller-style) acceptance condition 
over the \emph{orbits} (not the states!) of automata. By doing so, it turns out that our languages not only are finite-memory, but they retain computational properties, such as closure under boolean operations and decidability of emptiness (thus, containment and equivalence), which we prove by providing finite representations, and effective constructions. As in the case of standard $\omega$-automata, the shift to infinite words requires these results to be proved from scratch, as it is not possible to merely extend proofs from the finite words case. These results enable automata-theoretic model checking to be performed on systems with infinite resources, using traditional model-checking algorithms. 

Furthermore, we prove that the defined languages are determined by their ultimately-periodic fragments. This theorem is fundamental for \emph{learning} logical properties (see e.g., \cite{MP95}, or \cite{FCCTW08}) and has been used to provide a complete minimisation procedure for equivalent representations of Muller automata \cite{CV12}. Establishing this theoretical result in the nominal case is an important step towards the application of such techniques to global computing scenarios.
%
%\section{Example: peer-to-peer system}\label{sec:example}
%\input{example}
%
%\section{Background}\label{sec:background}%!TEX root=ndma.tex
\paragraph{Notation.} 
Throughout the paper: $f \colon X \to Y$ is a total function, $f \colon X \inj Y$ is total and injective, $f \colon X \pto Y$ is partial; $\dom(f)$ is the subset of $X$ where $f$ is defined, $\Im(f)$ its image. Symbol $\omega$ denotes the set of natural numbers. For $s$ a sequence, we let $s_i$ or $s(i)$ denote its $i^{\mathit{th}}$ element. $R^*$ is the symmetric, transitive, reflexive closure of binary relation $R$. We use $\circ$ for (partial) function composition and also for ``relational'' composition, as usual, by seeing functions as relations.

We shall now briefly introduce nominal sets; we refer the reader to \cite{GP02} for more details on the subject. We assume a countable set of \emph{names} $\names$, and we write $\Perm$ for the group of finite-kernel permutations of $\names$, namely those bijections $\pi \colon \names \to \names$ such that the set $\{ a \mid \pi(a) \neq a \}$ is finite.
\begin{definition}
A \emph{nominal set} is a set $X$ along with an action for $\Perm$, that is a function $\cdot \colon \Perm \times X \to X$ such that, for all $x \in X$ and $\pi,\pi' \in \Perm$, $id_\names \cdot x = x$ and $(\pi \circ \pi') \cdot x = \pi \cdot (\pi' \cdot x)$. Also, it is required that each $x \in X$ has \emph{finite support}, meaning that there exists a finite $S \subseteq \names$ such that, for all $\pi \in \Perm$, $\restr{\pi}{S} = id_S$ implies $\pi \cdot x = x$. We denote the least\footnote{It is a theorem that whenever there is a finite support, there is also a least support.} such $S$ with $\supp(x)$. An \emph{equivariant function} from nominal set $X$ to nominal set $Y$ is a function $f : X \to Y$ such that, for all $\pi$ and $x$, $f(\pi \cdot x) = \pi \cdot f(x)$.
\end{definition}
%
\begin{definition}
Given $x \in X$, the \emph{orbit} of $x$, denoted by $\orb(x) $, is the set $\{ \pi \cdot x \mid \pi \in \Perm\} \subseteq X$. For $S \subseteq X$, we write $\orb(S)$ for $\{ \orb(x) \mid x \in S\}$. We call $X$ \emph{orbit-finite} when $\orb(X)$ is finite.
\end{definition}

\noindent Note that $\orb(X)$ is a partition of $X$. The prototypical nominal set is $\names$ with $\pi \cdot a = \pi(a)$ for each $a \in \names$; we have $\supp(a) = \{a\}$, and $\orb(a) = \names$.

%
%\section{Nominal \texorpdfstring{$\omega$}{omega}-regular languages}\label{sec:languages}\input{languages}
%
%
%\section{Finite automata}\label{sec:hd-automata}
%\input{hd-automata}
%
%
%\section{Synchronized product and boolean operations}\label{sec:sync-product}
%\input{synchronized-product}
%%
%%!TEX root=ndma.tex
\newcommand{\compl}[1]{\overline{#1}}
%
Let $\Lang_1$ and $\Lang_2$ be $\omega$-regular nominal languages, and let $\autom_1 = (\tstr_1,\acc_1)$  and $\autom_2 = (\tstr_2,\acc_2)$ be automata for these languages, where $\tstr_1$ and $\tstr_2$ are the underlying transition structures. By \cref{thm:inf-correspondence} above, 
we are now able to show that constructing the automaton for a boolean combination of $\Lang_1$ and $\Lang_2$ amounts to defining an appropriate accepting set for $\tstr_1 \syncp \tstr_2$.


\begin{theorem}
Using the transition structure $\tstr_1 \syncp \tstr_2$, define the accepting conditions $\acc_\cap = \{ S \subseteq \syncQ \mid \pi_1(S) \in \acc_1 \land \pi_2(S) \in \acc_2 \}$, $\acc_\cup = \{ S \subseteq \syncQ \mid \pi_1(S) \in \acc_1 \lor \pi_2(S) \in \acc_2 \} $ and $\acc_{\compl{\Lang_1}} = \Pow(Q_1) \setminus \acc_1$, where $Q_1$ are the states of $\autom_1$. The obtained \hdma{}s accept, respectively, $\Lang_1 \cap \Lang_2$, $\Lang_1 \cup \Lang_2$, and $\compl{\Lang_1}$.
\label{thm:bool-closure}
\end{theorem}
%
\begin{theorem}
\label{thm:decidable}
Emptiness and, as a corollary, equality of $\omega$-regular nominal languages are decidable.
\end{theorem}

%
%\section{Ultimately-periodic words}\label{sec:up-words}%!TEX root=ndma.tex
%
%
%
An \emph{ultimately periodic} word is a word of the form $uv^\omega$, with $u,v$ finite words. Given a language of infinite words $\Lang$, let $UP(\Lang)$ be its \emph{ultimately periodic fragment} $\{ \alpha \in \Lang \mid \alpha = uv^\omega \land u,v \; \text{are finite} \}$. It has been proven in \cite{CalbrixNP93,Buchi62} that, for every two $\omega$-regular languages $\Lang_1$ and $\Lang_2$, $UP(\Lang_1) = UP(\Lang_2)$ implies $\Lang_1 = \Lang_2$, i.e.\ $\omega$-regular languages are characterised by their ultimately periodic fragments.
In this section we aim to extend this result to the nominal setting. 

The preliminary result to establish, as in the classical case, is that every non-empty nominal $\omega$-regular language $\Lang$ contains at least one ultimately periodic word. For $\omega$-regular languages, this involves finding a loop through accepting states in the automaton and iterating it. For \hdmas, freshness constraints could forbid consuming the same name in consecutive traversals of the same transition.
We first show that, given a loop in a \hdma{}, there always is a path induced by consecutive traversals of the loop, such that its initial and final configurations coincide. Thus, such path can be taken an arbitrary number of times.

Fix a loop $L \;:=\; p_0 \htr{l_0}{\sigma_0} p_1 \htr{l_1}{\sigma_1} \dots \htr{l_{n-1}}{\sigma_{n-1}} p_0$ (the specific \hdma{} is not relevant). We write $\ul{i}$ for $i \mod n$. For all $i=0,\dots,n-1$, let $\widehat{\sigma}_i \colon \weight{p_\ul{i+1}} \pto \weight{p_i}$ be the partial maps telling the history of old registers and ignoring the new ones, formally $\widehat{\sigma}_i := \sigma_i \setminus \{ (x,y) \in \sigma_i \mid y = \star \}$, and let $\widehat{\sigma} \colon \weight{p_0} \pto \weight{p_0}$ be their composition $\widehat{\sigma}_0 \circ \widehat{\sigma}_1 \dots \circ \widehat{\sigma}_{n-1}$. We define the set $I$ as the greatest subset of $\dom(\widehat{\sigma})$ such that $ \widehat{\sigma}(I) = I$,
i.e.\ $I$ are the registers of $p_0$ that ``survive'' along $L$. We denote by $T$ all the other registers, namely $T := \weight{p_0} \setminus I$. These are registers whose content is eventually discarded (not necessarily within a single loop traversal), as the following lemma states.
%
%
\begin{lemma}
\label{lem:rho-forget}
Given any $x \in T$, let $\{x_j\}_{j \in J_x}$ be the smallest sequence that satisfies the following conditions: $x_0 = x$ and $x_{j+1} = \sigma_{\ul{j}}^{-1}(x_j)$, where $j+1 \in J_x$ only if $\sigma_{\ul{j}}^{-1}(x_j)$ is defined. Then $J_x$ has finite cardinality.
\end{lemma}
%
Now, consider any assignment $\rrho_0 \colon \weight{p_0} \to \names$. We give some lemmata about paths that start from $(p_0,\rrho_0)$ and are induced by consecutive traversals of $L$. The first one says that the assignment for $I$ given by $\rrho_0$ is always recovered after a fixed number of traversals of $L$, regardless of which symbols are consumed. In the following, given a sequence of transitions $P$, we write $(q_1,\rho_1) \TrP{v}{P} (q_2,\rho_2)$ whenever $(q_1,\rho_1) \Tr{v} (q_2,\rho_2)$ and such path is induced by $P$.
%
\begin{lemma} 
\label{lem:idI}
There is $\id \geq 1$ such that, for all $v_0,\dots,v_{\id-1}$ satisfying $(p_0,\rrho_0) \TrP{v_0}{L} (p_0,\rrho_1) \TrP{v_1}{L} \dots \TrP{v_{\id-1}}{L} (p_0,\rrho_\id)$ we have $\restr{ \rrho_\id }{I} = \restr{ \rrho_0 }{I}$.
\end{lemma}
%
The second one says that, after a minimum number of traversals of $L$, a configuration can be reached where the initial values of $T$, namely those assigned by $\rrho_0$, cannot be found in any of the registers.
%
\begin{lemma}
There is $\forg \geq 1$ s.t., for all $\gamma \geq \forg$ ,there are $v_0,\dots,v_{\gamma-1}$ satisfying $(p_0,\rrho_0) \TrP{v_0}{L} (p_0,\rrho_1) \TrP{v_1}{L} \dots \TrP{v_{\gamma-1}}{L} (p_0,\rrho_\gamma)$, with $\Im(\rrho_\gamma) \cap \rrho_0(T) = \varnothing$.
\label{lem:forgetT}
\end{lemma}
%
%
We give the dual of the previous lemma: if we start from a configuration where registers are not assigned values in $\rrho_0(T)$, then these values can be assigned back to $T$ in a fixed number of traversals of $L$, regardless of the initial assignment.

\begin{lemma}
There is $\ass \geq 1$ such that,
for any $\trho_0 \colon \weight{p_0} \to \names$ with $\Im(\trho_0) \cap \rrho_0(T) = \varnothing$, there are $v_0,\dots,v_{\ass-1}$ satisfying $ (p_0,\trho_0) \TrP{v_0}{L} (p_0,\trho_1) \TrP{v_1}{L} \dots \TrP{v_{\ass-1}}{L} (p_0,\trho_\ass)$, with $\restr{\trho_\ass}{T} = \restr{\rrho_0}{T}$.
\label{lem:initT}
\end{lemma}
%
%
Finally, we combine the above lemmata. We construct a path where: (1) the values assigned to $T$ are forgotten and then recovered (2) the values assigned to $I$ are swapped, but the initial assignment is periodically regained. Therefore, the length of such path should allow (1) and (2) to ``synchronize'', so that the final assignment is again $\rrho_0$.

\begin{theorem}
\label{thm:loop}
%
For each loop $L$ with initial state $p_0$, and assignment $\rrho_0 \colon \weight{p_0} \to \names$, there are $v_0,\dots,v_n$ such that $(p_0, \rrho_0) \TrP{v_0}{L} (p_0, \rrho_1) \TrP{v_1}{L} \cdots \TrP{v_n}{L} (p_0,\rrho_0) $.
\end{theorem}
%
\begin{example} We justify the construction on the \hdma{} of \cref{fig:upwords-ex}, with initial assignment $\rho_0(x_0) = a$,$\rho_0(y_0) = b$ and $\rho_0(z_0) = c$. Consider the loop $L$ formed by all the depicted transitions. We have $I = \{x_0,y_0\}$ and $T = \{z_0\}$. Look at the path $(q_0,[\subs{a}{x_0},\subs{b}{y_0},\subs{c}{z_0}]) \tr{c} (q_1,[\subs{b}{x_1},\subs{a}{y_1},\subs{c}{z_1}]) \tr{d} (q_2,[\subs{b}{x_2},\subs{a}{y_2},\subs{d}{z_2}])
	\tr{b} (q_0,[\subs{b}{x_0},\subs{a}{y_0},\subs{d}{z_0}])$ where $d \neq a,b,c$. The values of $x_0$ and $y_0$ are swapped according to the permutation $(a \; b)$, and $d$ is assigned to $z_0$. Our aim is to recover $\rho_0$ again. According to \cref{lem:idI}, $x_0$ and $y_0$ get their assignment back in $\theta = 2$ traversals of $L$ (in fact $(a\; b)^2 = (a\; b)$). As for $z_0$, its assignment is established in the second transition, but $c$ should not have been assigned to any register of $q_1$ in order for it to be consumed during this transition. This is where \cref{lem:forgetT} comes into play: it says that in at least $\epsilon = 1$ traversals of $L$ the name $c$ is discarded. This is exactly what happens in the path shown above. Then we can assign $c$ to $z_0$ in another $\zeta = 1$ traversal of $L$, according to \cref{lem:initT}. Since $\epsilon + \zeta  = \theta = 2$, traversing $L$ twice is enough (e.g., consider the path $cdbdca$).
\end{example}
%
\noindent Finally we introduce the main results of this section.
%
\begin{theorem}
\label{thm:up-fragment}
When $\Lang$ is a non-empty nominal $\omega$-regular language, $UP(\Lang) \neq \emptyset$.
\end{theorem}

\begin{theorem}
\label{thm:up-determinacy}
For $\Lang_1,\Lang_2$ nominal $\omega$-regular, $UP(\Lang_1) = UP(\Lang_2) \implies\Lang_1 = \Lang_2$.
\end{theorem}
% 
Note that a similar result could not be achieved in the presence of so-called \emph{global freshness} \cite{Tze11}, e.g.\ the one-state automaton accepting only globally fresh symbols would have empty ultimately periodic fragment, just like the empty language. As a concluding remark, we note that, by \cref{thm:up-determinacy}, every $\omega$-regular language is characterized by a sublanguage of finitely supported words (the support of $uv^\omega$ just contains the finitely many symbols in $u$ and $v$). We find this result appealing, given the central role of the notion of support in the nominal setting. 
%
%\section{Conclusions}\label{sec:conclusions}
%%!TEX root=ndma.tex
This work is an attempt to provide a simple definition that merges the theories of nominal automata and $\omega$-regular languages, retaining effective closure under boolean operations, and decidability of emptiness, and language equivalence. We sketch some possible future directions. A very relevant application of formal verification in the presence of fresh resources could be model-checking of nominal process calculi. However, the presented theory only accommodates the deterministic case; undecidability issues arise for non-deterministic systems. Future work will be directed to identify (fragments of) nominal calculi that retain decidability. For this, one needs to limit not only non-determinism, but also parallel composition (again, decidability may be an issue otherwise). A calculus that could be handled by the current theory is a deterministic, finite-control variant of the $\pi$-calculus; capturing analogous versions of more recent calculi, e.g., $\psi$-calculi \cite{Bengtson11}, should be possible, as they are based on nominal structures with notions of permutation action, support, orbits. As mentioned in \autoref{sec:example}, we argue that deterministic behavior is enough to specify meaningful policies. 
%
Furthermore, recall that automata correspond to logic formulae: \hdmas{} could be used to represent logic formulae with binders; it would also be interesting to investigate the relation with first-order logic on nominal sets \cite{Bojanczyk13}. There may be different logical interpretations of \hdmas, where causality or dependence \cite{Vnnen07,Galliani12} between events are made explicit. Finally, extending the two-sorted coalgebraic representation of Muller automata introduced in \cite{CV12} to \hdmas{} would yield canonical representative of automata up to language equivalence.

%%!TEX root=ndma.tex
\emph{Related work.} Automata over infinite data words have been introduced to prove decidability of satisfiability for many kinds of logic: LTL with freeze quantifier \cite{DemriL09}; safety fragment of LTL \cite{Lazic11}; $FO$ with two variables, successor, and equality and order predicates \cite{BojanczykDMSS11}; EMSO with two variables, successor and equality \cite{KaraST12}; generic EMSO \cite{Bollig11}; EMSO with two variables and LTL with additional operators for data words \cite{KaraT10}. The main result for these papers is decidability of nonemptiness. These automata are ad-hoc, and often have complex acceptance conditions, while we aim to provide a simple and seamless nominal extension of a well-known class of automata. We can also cite variable finite automata (VFA) \cite{GrumbergKS10}, that recognize patterns specified through ordinary finite automata, with variables on transitions. Their version for infinite words (VBA) relies on B\"uchi automata. VBA are not closed under complementation and determinism is not a syntactic property. For our automata, determinism is easily checked and we have closure under complementation. On the other hand, VBA can express ``global'' freshness, i.e.\ symbols that are different from all the others. 
%
%\paragraph{Acknowledgements.} The authors thank Nikos Tzevelekos, Emilio Tuosto and Gianluca Mezzetti for several fruitful discussions related to nominal automata.

\bibliographystyle{splncs}
\bibliography{bibliography}

\end{document}
