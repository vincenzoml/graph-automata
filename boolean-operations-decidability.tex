%!TEX root=ndma.tex
\newcommand{\compl}[1]{\overline{#1}}
%
Let $\Lang_1$ and $\Lang_2$ be $\omega$-regular nominal languages, and let $\autom_1 = (\tstr_1,\acc_1)$  and $\autom_2 = (\tstr_2,\acc_2)$ be automata for these languages, where $\tstr_1$ and $\tstr_2$ are the underlying transition structures. By \cref{thm:inf-correspondence} above, 
we are now able to show that constructing the automaton for a boolean combination of $\Lang_1$ and $\Lang_2$ amounts to defining an appropriate accepting set for $\tstr_1 \syncp \tstr_2$.


\begin{theorem}
Using the transition structure $\tstr_1 \syncp \tstr_2$, define the accepting conditions $\acc_\cap = \{ S \subseteq \syncQ \mid \pi_1(S) \in \acc_1 \land \pi_2(S) \in \acc_2 \}$, $\acc_\cup = \{ S \subseteq \syncQ \mid \pi_1(S) \in \acc_1 \lor \pi_2(S) \in \acc_2 \} $ and $\acc_{\compl{\Lang_1}} = \Pow(Q_1) \setminus \acc_1$, where $Q_1$ are the states of $\autom_1$. The obtained \hdma{}s accept, respectively, $\Lang_1 \cap \Lang_2$, $\Lang_1 \cup \Lang_2$, and $\compl{\Lang_1}$.
\label{thm:bool-closure}
\end{theorem}
%
\begin{theorem}
\label{thm:decidable}
Emptiness and, as a corollary, equality of $\omega$-regular nominal languages are decidable.
\end{theorem}
